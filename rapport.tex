\documentclass[11pt,french,a4paper]{report}
\usepackage[utf8]{inputenc}

\usepackage{geometry}
\geometry{a4paper}

\title{Rapport de Stage fin de DUT - SuperBeeLive}
\author{Olivia SERENELLI-PESIN}

\begin{document}

\maketitle

\clearpage
\newpage 

\chapter*{Remerciements}

J’adresse mes remerciements aux personnes qui m’ont permis de réaliser ce stage dans l’équipe de SuperBeeLive. \\
Tout d’abord Matthieu ROUSSET, initiateur du projet SuperBeeLive à l'IBMM (Insitut Biomoléculaire Max Mousseron) 
qui m’a accueilli au sein de son équipe. \\
Ensuite, Capucine CARLIER pour ses nombreuses explications biologiques sur les abeilles ainsi que sa disponibilité 
afin de comprendre au mieux les enjeux et les besoins des biologistes pour mon projet. \\
Enfin, Sebastien DRUON pour m’avoir encadré, aidé à m’intégrer dans le milieu de la recherche et aidé sur une multitude 
de sujets, aussi bien du point de vu universitaire que sur les tâches qui m’ont été confiées. \\

\tableofcontents

\clearpage

\chapter{Introduction : Présentation du projet de recherche}
\section{La structure d'accueil} 

Le projet étant réalisé par plusieurs laboratoires de recherches, j'ai dû évoluer au seins de différentes organisations. 
Tout d'abord il y a mon employeur, l'Université de Montpellier, qui englobe d'autres structures où j'ai pu évoluer. 
L'université a été, pour moi, une entitée administrative. \\ 
Avec elle, le CNRS (Centre National de Recherche Scientifique) accueille dans ses locaux le rucher expérimental où 
nous pouvons effectuer nos tests d'installation pour le projet. J'ai pu m'y rendre plusieurs fois afin d'observer les abeilles
et voir le travail déjà effectué et évoluer au fur et à mesure. C'est également chez eux que nous aurons un serveur d'installé.\\
Ensuite, l'IBMM est le laboratoire qui a engagé l'argent lié au projet afin de pouvoir me recrutrer lors de ce stage. 
C'est également une structure qui a été seulement administrative de mon point de vu puisqu'ils m'ont envoyés travailler dans 
les bureaux du LIRMM (Laboratoire d'informatique, de robotique et de microéléctronique de Montpellier), autre laboratoire de 
recherche, afin que je puisse être aux côtés de Sébastien DRUON qui m'aura donné la grande majorité de mes tâches à effectuer. 
J'ai pu y avoir mon bureau en face de M. DRUON, me permettant d'avoir une certaine autonomie mais aussi de pouvoir faire
appel à lui facilement lorsque j'en avais besoin.\\
\\
C'est dans ce contexte de recherche que j'ai pu découvrir et travailler sur le projet SuperBeeLive. \\ 


\section{Projet SuperBeeLive}
\subsection{Le Global} 
La santé et le développement des abeilles est aujourd’hui une question de plus en plus étudiée. Les bouleversements
majeurs de notre planète et de l’activité humaine se traduisant par une augmentation alarmante de la mortalité
des colonies et une chute de la production du miel dans nos pays développés, il est urgent de se préoccuper de leur futur. 
La situation des abeilles domestiques alerte le pouvoir public sur l’accélération de la dégradation de la biodiversité des 
pollinisateurs domestiques et sauvages, et de la flore qui en dépend. Ces dégâts sont dûs, entre autre, à l’apparition 
et la prolifération d’espèces invasiges pour les abeilles, provoquant maladies et déteriorations. 

Le projet consiste en la structuration de plusieurs collaborations existantes ou nouvelles autours du developpement 
d’une ruche plate instrumentée destinée au monitorage détaillé de la santé de l’abeille et des écosystèmes. Son but est de
pouvoir répondre à des questions clés, notamment autours des mécanismes physiopathologiques et des maladies chroniques 
dûes au parasites ainsi qu’aux altérations de l’écosystèmes et des qualités nutritives des produits des ruches. \\
Répondre à ces questions permettra de regrouper différentes solutions technologiques systèmatiques, 
automatiques et non invasives à la collection de données usuelles déterminantes dans chacun des thèmes abordés.
Les différentes travaux déjà effectués autours de ce sujet ne visaient qu’un seul type de problème à la fois, 
ne permettant pas une vision globale des difficultées rencontrées par les abeilles. Notre but est de réunir les différentes
données qui peuvent être utilisés pour étudier l’influence des éléments et événements extérieurs sur leur santé et leur cadre de vie.\\

Concrétement, l'équipe de SuperBeeLive va concevoir une ruche plate afin d'y mettre en place plusieurs type de capteurs
(hygrométrie, vibrations, température interne et externe, etc) ainsi que des caméras qui filmeront l'intérieur et 
l'extérieur de la ruche. \\

%Schéma prévisionnel de la ruche, photos TODO

Cette instrumentation nous permettra dans un premier temps d'observer les abeilles afin de visualiser leurs comportements 
(danse, regroupement, protection, etc) ainsi que les parasites comme les frelons asiatiques ou les varroas. %TODO lien doc varoa/frelon 
Une fois ces observations faites et documentées, il y aura assez de matière pour pouvoir créer des algorithmes qui reconnaitront 
automatiquement ces comportements pour en sortir des informations spécifiques.\\
Par exemple, un des buts et de repérer la danse d'une abeille qui permet aux autre d'indiquer où se trouve du polen et d'en
extraire les informations qu'elle transmet (la direction à prendre, le temps à parcourir pour y arriver etc).\\
%lien avec document sur les danses ? TODO
Avec ces recherches, il serait possible de mettre en place une vitrine web des caméras streamées et commentées 
automatiquement en temps réel, permettant à d'autres chercheurs d'avoir une ressource permanente pour travailler mais aussi
au grand public d'avoir un accès plus restreint mais pédagogique sur le comportement des abeilles. \\

\subsection{Mon rôle durant le stage}
Pour ce stage, plusieurs missions m'ont été affectées, une qui est devenue ma principales et d'autres, plus courtes dans le temps.\\

Comme dit plus haut, le but premier de l'installation des caméras sur la ruche est de pouvoir observer et annoter des vidéos manuellement
pour ensuite rendre cette tâche automatique. Seulement, ces annotations doivent être encadrées afin d'éviter que des données ne se perdent, 
et que les outils utilisés pour le faire ne soient pas les même d'une personne à une autre nous donnant alors des fichiers non 
uniformes et plus difficiles et long à traiter une fois réunis. \\
Ainsi, nous avons entreprit de créer un logiciel d'annotation, permettant de regarder en direct les caméras et de sauvegarder des 
morceaux de vidéos afin de pouvoir dessiner simplement dessus (encercler, mettre une flèche, encadrer, etc) et écrire quelques mots
sur ce qu'on y observe. Ces vidéos seraient sauvegardées dans un fichier contenant la vidéo et les annotations et pourront être 
visualisés de nouveau dans ledit logiciel mais aussi dans un lecteur plus classique mais sans les annotations. \\
Au final, celui-ci alégera le travail des biologiste, qui auront un outils sur mesure pour annoter les vidéos, mais aussi de M. Druon, 
pour qui il sera plus simple de récupérer et traiter les données.\\

Il est évident que dans un tel projet, beaucoup de données seront transmises et stockées. Ainsi, toute une partie d'administration 
système est à gérer. Dans notre cas, nous avons deux tâches importantes.\\
D'abord, la question du stockage des données commençait à se poser lors de mon arrivée en stage. Il fallait choisir, acheter, installer
puis configurer un serveur de stockage dans la salle serveur du CNRS.\\
Ensuite, le rucher du CNRS où notre ruche expérimentale est installé n'a pas de configuration réseau déjà établie. 
Une connexion par fibre optique est prévue, mais la suite de l'installation devra être gérée par nous même. Comme pour le serveur, 
il faudra choisir, installer et configurer un switch dans le rucher.\\
Pour ces deux tâches, la même problématique est soulevée : il faudra penser au grand nombre de données qui devront transiter sur le 
réseau et donc prévoir du matériel adapté.\\
D'autres réalisations auraient pû m'être affectées, comme le dimensionnement des caméras ou la construction des cartes éléctroniques
qui seront installées au centre de la ruche. Seulement, ces sujets s'éloignant de mon DUT Réseaux et Télécommunications et mes 
compétences et connaissances dans ces deux domaines étant limitées, je n'ai pas eu à travailler dessus.\\
Cependant, la possibilité d'un apprentissage lors d'école d'ingénieurs en systèmes embarqués a été évoquée, ce qui correspondrait 
plus à ces tâches.\\


\section{L'équipe}
%TODO organigrame 

Brouillon organigrame 

CNRS : Jean-Baptiste THIBAUD 50\%
IBMM : Matthieu ROUSSET 50\%
LMGC : Delphine JULIEN
       Capucine CARLIER
LIRMM : Jean TRIBOULET 
        Sébastien DRUON 
IUT de Béziers : Philiphe PUJAS -> stockage des données 
Equipe Bee@Wur Université Wegeningen (Néerlandais) -> ouverture internationnalle. 


\chapter{Projet principal}

    \section{Analyse du besoin et des fonctionnalités exigées} 
Dans le cadre de la récolte et de l'analyse des données, il était primmordial pour l'équipe d'avoir l'outils d'annotation décrit plus haut. \\
Celui-ci devait permettre de : \\
\begin{itemize}
    \item Visualiser les caméras en direct
    \item Démarrer la capture vidéo à tout moment 
    \item Ajouter un système de tag par mots clés afin de pouvoir trier facilement les vidéos 
    \item Avoir plusieurs types d'annotations (entourer, marquer, avoir des mouvements etc) 
    \item Retrouver toutes les mesures de la ruches (température, horaires, numéro de caméra, de ruche, de cadre etc )  
    \item Avoir un principe d'auteur 
    \item Pouvoir revisualiser les vidéos 
    \item Pouvoir remodifier les vidéos 
\end{itemize} \\
Afin de répondre à ces besoins, nous avons imaginé l'interface suivante : 
%TODO Faire le schéma de l'interface
Bien sûr, celle-ci a connu beaucoup d'évolutions au cours de son développement : au fur et à mesure de l'avancement, nous nous rendions compte
de certaines éventualités que nous n'avions pas imaginé et que nous avons intégré à la volée.  \\

    \section{Contraintes} 
Le langage de départ pour coder l'interface et ses fonctionnalités, que nous pourrons familiairement nommer partie moteur et partie physique, m'a été imposé. \\
C'est donc en C que j'ai dû réflechir à comment préparer et lier ces deux parties. Ce langage a été choisit tout simplement parceque c'est 
le langage que M. DRUON a pour habitute d'utiliser. \\
La partie moteur peut être développée en C sans trop d'ajout de librairies annexes autres que celles dites basiques (stdlib, stdio). Cependant, 
nous avons dû utiliser une librairie pour développer la partie physique. Nous avions le choix entre GTK ou QT. QT devant être utilisé en C++, 
notre choix s'est naturellement dirigé vers GTK 3.0. \\
Comme dans tout projets collaboratifs, la convergence des données est importante et peut parfois se réveler difficile à mettre en place. 
Heureusement pour nous, en programmation l'outil GIT est excellent pour travailler à plusieurs. L'ayant déjà vu en cours cette année, j'ai pu l'utiliser 
concrétement lors de mon stage. Cependant, avant de commencer à utiliser concrètement l'outil, il a été préférable que je me remette à niveau sur celui-ci
et ai donc entammé la lecture du livre Mastering Git %TODO VERIFIER LE TITRE
qui m'a été d'une grande aide pour avoir des bases solides. Pour héberger notre code, nous avons choisi de le déposer sur GitHub : il est donc accessible au 
public sous le nom superbeelive. 

Pour cela, le livre %TODO Integrer ref du livre 
m'a servi de référence de base quant à la manière de manier les éléments. Je me suis également aidée de divers tutoriels sur internet, mais surtout 
de la document officielle de GTK. %TODO Lien biblio 
\\
Souvent, on a tendance à utiliser Glade, un logiciel permettant de créer une interface GTK avec des outils graphiques. Dans mon cas, je m'en suis surtout servie 
en tant que bibliothèque pour voir et tester certains Widgets ainsi que leur fonctionnement. Cet utilitaire, certes plus rapide à prendre en mains, est lié 
au fait qu'il utilise des fichiers XML pour décrire l'interface ce qui, au final, complique la portabilité de l'interface d'une version du logiciel à l'autre. 
De plus, j'ai préféré apprendre à utiliser GTK en ligne de code "brut". Plus fastidieu au départ, j'ai trouvé cela plus simple à la longue : je maîtrisais vraiment 
mes outils et apprenais plus rapidement à utiliser un nouvel élément de la librairie.  
%TODO : intégrer capture d'écran glade ? 

        \subsection{Principe de GTK}
Apprendre à utiliser GTK sera plus long si la programmation objet nous est inconnu. En effet, même si le C n'est pas un langage objet, GTK lui, reprend 
les principes de la programation objet. Si vous êtes déjà familier avec ce type de langage, alors l'apprentissage sera bien plus rapide.  
L'idée principale de GTK est que l'on va ranger tous les éléments en fonction d'un autre élément, pour finir avec des sortes de poupées russes qui s'emboîtent 
et se rangent côte à côte pour donner un résultat final. Il exite une multitude d'éléments utilisables, appelés les Widgets. L'élément de base widgets contient des propriétés de 
base et, pour créer d'autre type d'élément, les développeurs de GTK y ont ajoutés d'autres propriétés à celles déjà existantes. 
Au fur et à mesure, ces ajouts de propriétés ont permit de créer toute sorte de Widget : des textes, des labels, des tableaux, des séparateur, des listes, des bouttons... 
Ces héritage et cette hierarchie des objets va nous construire un arbre d'élements, nous permettant de savoir ce que l'on peut faire avec nos Widgets : 
%TODO image de la hierarchie
En effet, si par exemple je veux changer un paramètre d'un élément "GtkSpinButton", et que je ne trouve pas de fonction affectant directement ce type d'élément, 
alors je vais regarder si chez ses parents je peux trouver le paramètre voulu, à savoir "GtkEntry", et "GtkWidget". \\

Chaque paramètre disponible pour les Widgets peut être changé directement à l'aide de fonctions retrouvable dans l'excelente documentation en ligne de GTK 3.0. 
Il est totalement déconseiller de changer directement la valeur des paramètre à la mains en utilisant, par exemple des pointeurs. Il faut partir du principe que 
chaque élément a son nombre de fonctions associées et que tout est prévu pour pouvoir faire ce que l'on veut sur notre interface. \\ 
\\

        \subsection{Des boites, dans des boites...}
Une fois la manière de créer les éléments comprise, il faut ensuite comprendre comment les ranger. D'abord, on va créer une fenêtre, soit une GtkWindows. Dans celle-ci, on 
ne peut y poser qu'un seul Widget. C'est pour ça que nous avons les éléments "GtkContainers" : c'est avec eux que nous allons pouvoir structurer et placer tous les autres 
éléments dans notre fenêtre. Le container de base est la box. A sa création, nous devons simplement indiquer si celle-ci va être horizontale ou verticale, c'est à dire
si les éléments que nous allons mettre dedans vont être placé de haut en bas ou de droite à gauche. Une fois cette première GtkBox placée, le problème de limitation de place 
imposée par la GtkWindow n'en est plus un. Maintenant, nous pouvons ranger ce que nous voulons dans cette boite. C'est là que le système de poupée russe prend son sens : 
nous allons devoir jouer avec les différentes boîtes pour créer les espaces que nous désirons. Voici le schéma de construction que j'ai fais pour la première fenêtre de 
l'application : 
%TODO intégrer schéma de boite de la fenetre n°1
On peut voir que la GtkBox box\_principale prend toute la place de ma fenêtre GtkWindow. Dans cette box\_principale, j'y ai rangé verticalement box\_up et box\_down. Dans
box\_down j'ai integré box\_left et box\_right horizontalement, me permettant d'avoir quatre zones délimitées. Ensuite, j'ai créé des box pour mes éléments plus précis :
dans box left j'ai directement intégré box\_video dans laquelle se trouve la vidéo. Dans box right on va retrouvé box\_info, box\_meta, box\_btn\_cam, box\_btn\_video, 
box\_info\_time et box\_file. Chacune de ces box m'ont permit de ranger à l'intérieur les éléments que je voulais retrouver : les boutons, les textes, les descriptions etc. 
A noter qu'en plus de ces boîtes j'ai également parfois ranger des séparateur, les barre horizontale et verticale, permettant d'ajouter un peu de structure visuelement. \\
Une fois que l'on sait comment ranger les éléments sur une fenêtre et quel est le principe de ces éléments dans l'idée, il ne manque plus qu'à voir comment concretement, 
dans le code, on applique ces principes. \\   
GTK nécessite une organisation rigoureuse si on veut pouvoir s'y retrouver et imaginer facilement comment est construit la fenêtre décrite dans le code. 
C'est pourquoi j'ai préféré appliquer une manière de classer les éléments qui me permette de retrouver facilement ce avec quoi je veux travailler dans l'instant. \\
Ainsi, j'ai voulu identifier les différentes étapes de la création d'un Widget et les séparer en trois grandes parties pour m'y retrouver plus facilement. \\

        \subsection{Création et Définition}
La première partie est la déclaration d'un Widget et sa définition. La création est simplement la déclaration de la variable avec son type. A noter que j'utilise des pointeurs
pour toute mon application et que mes exemples viennent directement des fichiers win\_main.c et win\_main.h. La structure et l'explication de ces pointeurs est décrite
dans la partie "Les structures" de "Définition des fonctionnalités". \\
Créons une fenêtre, une box et un label : \\
%TODO trouver latex qui permet de faire des ligne de commandes
--> Dans win\_main.h : 
GtkWidget* window ; \\ 
GtkWidget* box ;  \\ 
GtkWidget* label ; \\ 
Pour ces trois exemples, ces éléments sont tous du type GtkWidget. Il arrive que, dans de rare cas, il soit nécessaire de déclarer un autre type d'objet. \\ 

Ces trois éléments existent désormais, maintenant il faut les définir, c'est à dire expliquer de quel type de Widget il seront et leur donner leur propriétés. \\
--> Dans win\main.c : 
tmp->window = gtk\_window\_new (GTK\_WINDOW\_TOPLEVEL) ;
Ici, on utilise la fonction "gtk\_window\_new" pour indiquer que "window" est une fenêtre qui s'ouvre en premier plan avec l'argument "GTK\_WINDOW\_TOPLEVEL". \\

tmp->box\_principal = gtk\_box\_new( GTK\_ORIENTATION\_VERTICAL, 0) ;
Comme dit plus haut, à la création d'une box on indique d'abord si on veut que les éléments soient posés verticalement ou horizontalement. Ensuite, le second
argument permet de définir le nombre de pixel d'écart par défaut entre les éléments, ici 0px, donc ils seront collés l'un à l'autre.  \\
%TODO VERIFIER ARGUMENT  !!
tmp->label = gtk\_label\_new("Je suis le texte du label") ; \\
Enfin, le label nécessite lui aussi un argument, ici ce sera le texte noté de base. On peut laisser ce texte vide si on ne rempli pas les "".  \\

Les exemples ci-dessus ont tous eu besoin d'une définition précise du Widget, avec à chaque fois au moins un paramètre à régler. Il arrive parfoit que la création
ne nécéssite aucun paramètre, et que des modification doivent être apportées ensuite. Par exemple, si je veux que mon label ai une écriture plus stylisée, je peux
utiliser :  \\
gtk\_label\_set\_markup(GTK\_LABEL(tmp->label), "<span foreground=\"black\" font=\"10\"><b>Je suis le label stylisée avec du bold</b></span>"); \\
Ici, il faut préciser avant le nom de la variable à modifier quel type de widget on compte modifier. En effet, comme lors de la déclaration il est indiqué 
que label est de type GtkWidget et que la fonction que l'on veut utiliser ne s'applique que sur un GtkLabel, il est donc nécessaire d'indiquer que cette fois-ci, 
c'est de type un GtkLabel depuis la fonction plus haut à l'aide de "GTK\_LABEL". 
Cette fois ci, mon label ressemblera à ça : 
%TODO incorporer avant/après du label
\\
Ainsi, il existe plein de fonctions différentes permettant de régler ce genre de paramètres sur les widgets, et parfois il est nécessaire de faire appel à elle pour 
avoir un Widget correct. Il est intéressant d'explorer les fonctions existantes afin de personnaliser au mieux notre interface. \\

Une fois les Widgets créés, il faut maintenant les placer et indiquer qu'il faut les afficher. \\ 
 
\subsection{Placement, cosmétique et affichage} \\
Maintenant, nous allons appliquer la théorie qui avait été développée au dessus : les poupées russes. On va commencer par dire où va se placer la box : \\ 
 gtk\_container\_add (GTK\_CONTAINER (tmp->window), tmp->box) ; \\
Window est un GtkContainer, nous utilisons donc la fonction qui permet d'affecter un élément à un container. Le premier argument indique le container où nous
devons insérer un élément. Le second nous indique quel élément est à placer. \\

Ensuite, nous faisons de même pour la box : \\
  gtk\_box\_pack\_start(GTK\_BOX(tmp->box\_principal), tmp->label, FALSE, FALSE, 0) ; \\
Box étant de type GtkBox, nous utilisons cette fois la fonction qui lui est associée. Comme pour le Window, il faut indiquer la boîte concernée par 
l'élément à ajouter, mais cette fois-ci trois autres arguments sont à renseigner concernant la manière dont les Widgets vont se comporter à l'intérieur
de la box. Mais d'abord, il faut savoir que par défaut (donc que tout est sur FALSE), les box allouent la même place à tous les Widgets qu'ils contiennent
sans déborder sur le reste.\\
\begin{itemize}
    \item Expand : peut être TRUE ou FALSE. Permet aux widgets de prendre plus de place si besoin si réglé sur TRUE. \\ 
    \item Fill : peut être TRUE ou FALSE. Ne peut être actif que si Expand est aussi TRUE. Alloue la hauteur ou la largeur totale de la box, 
        celon si la box est une horizontale ou vertical. \\
    \item Padding : valeur numérique. Indique l'espacement en pixel (px) entre les widget que contient la box. \\ 
\end{itemize}

Les éléments sont maintenant placés. Afin d'avoir une relecture facile et pour savoir qui se trouve dans quelle box ou container, j'ai appliqué une indentation 
qui m'est propre mais que je trouve plus lisible, voici l'exemple pour win\_main : \\
%TODO insérer ex de win_main 
Ainsi, à chaque fois que je vais dans une sous box, j'indente avec une tabulation. Au final, le rendu me permet de voir qui est à quel niveau. \\

Une fois que les éléments sont placés dans les boîtes, ceux-ci ne sont par contre pas arrangés précisements. Il faut maintenant indiquer où il faut appliquer des marges, 
comment doivent se placer les éléments. Sachant que la première chose à faire pour cette étape est de jouer avec les paramètres EXPAND et FILL des box qui permettent, 
par exemple, d'autoriser ou non à un élément de s'agrandir en même temps que la fenêtre. \\
Voici un exemple de ce que donne une fenêtre sans la partie cosmétique et arrangement précis puis avec : \\ 
%image d'un avant après de main_win
Comme pour la création et le placement, nous allons utiliser des fonctions pour indiquer ce que l'on veut faire avec le Widget, en voici quelques exemples : \\


        \subsection{Signaux et fonctions}
    \section{Définition des fonctionnalités} 
        \subsection{Les structures}
        \subsection{Les vidéos}
    

\chapter{Acquis et compétences}
    \section{Missions annexes}
    \section{Compétences développées}

\chapter{Conclusion}

\chapter{Références}
    \section{Annexes}
    \section{Biblio}
    \section{Lexique}

\end{document}
