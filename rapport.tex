\documentclass[11pt, french]{report}
\usepackage[utf8]{inputenc}

\title{Rapport de Stage fin de DUT - SuperBeeLive}
\author{Olivia SERENELLI-PESIN}

\begin{document}

\maketitle

\clearpage
\newpage 

\section*{Remerciements}

J’adresse mes remerciements aux personnes qui m’ont permis de réaliser ce stage dans l’équipe de SuperBeeLive. \\
Tout d’abord Matthieu ROUSSET, initiateur du projet SuperBeeLive à l'IBMM (Insitut Biomoléculaire Max Mousseron) 
qui m’a accueilli au sein de son équipe. \\
Ensuite, Capucine CARLIER pour ses nombreuses explications biologiques sur les abeilles ainsi que sa disponibilité 
afin de comprendre au mieux les enjeux et les besoins des biologistes pour mon projet. \\
Enfin, Sebastien DRUON pour m’avoir encadré, aidé à m’intégrer dans le milieu de la recherche et aidé sur une multitude 
de sujets, aussi bien du point de vu universitaire que sur les tâches qui m’ont été confiées. \\

\tableofcontents
\clearpage

\section{Introduction : Présentation du projet de recherche}
\subsection{La structure d'accueil} 

Le projet étant réalisé par plusieurs laboratoires de recherches, j'ai dû évoluer au seins de différentes organisations. 
Tout d'abord il y a mon employeur, l'Université de Montpellier, qui englobe d'autres structures où j'ai pu évoluer. 
L'université a été, pour moi, une entitée administrative. \\ 
Avec elle, le CNRS (Centre National de Recherche Scientifique) accueille dans ses locaux le rucher expérimental où 
nous pouvons effectuer nos tests d'installation pour le projet. J'ai pu m'y rendre plusieurs fois afin d'observer les abeilles
et voir le travail déjà effectué et évoluer au fur et à mesure. C'est également chez eux que nous aurons un serveur d'installé.\\
Ensuite, l'IBMM est le laboratoire qui a engagé l'argent lié au projet afin de pouvoir me recrutrer lors de ce stage. 
C'est également une structure qui a été seulement administrative de mon point de vu puisqu'ils m'ont envoyés travailler dans 
les bureaux du LIRMM (Laboratoire d'informatique, de robotique et de microéléctronique de Montpellier), autre laboratoire de 
recherche, afin que je puisse être aux côtés de Sébastien DRUON qui m'aura donné la grande majorité de mes tâches à effectuer. 
J'ai pu y avoir mon bureau en face de M. DRUON, me permettant d'avoir une certaine autonomie mais aussi de pouvoir faire
appel à lui facilement lorsque j'en avais besoin.\\
\\
C'est dans ce contexte de recherche que j'ai pu découvrir et travailler sur le projet SuperBeeLive. \\ 


\subsection{Projet SuperBeeLive}
\subsubsection{Le Global} 
La santé et le développement des abeilles est aujourd’hui une question de plus en plus étudiée. Les bouleversements
majeurs de notre planète et de l’activité humaine se traduisant par une augmentation alarmante de la mortalité
des colonies et une chute de la production du miel dans nos pays développés, il est urgent de se préoccuper de leur futur. 
La situation des abeilles domestiques alerte le pouvoir public sur l’accélération de la dégradation de la biodiversité des 
pollinisateurs domestiques et sauvages, et de la flore qui en dépend. Ces dégâts sont dûs, entre autre, à l’apparition 
et la prolifération d’espèces invasiges pour les abeilles, provoquant maladies et déteriorations. 

Le projet consiste en la structuration de plusieurs collaborations existantes ou nouvelles autours du developpement 
d’une ruche plate instrumentée destinée au monitorage détaillé de la santé de l’abeille et des écosystèmes. Son but est de
pouvoir répondre à des questions clés, notamment autours des mécanismes physiopathologiques et des maladies chroniques 
dûes au parasites ainsi qu’aux altérations de l’écosystèmes et des qualités nutritives des produits des ruches. \\
Répondre à ces questions permettra de regrouper différentes solutions technologiques systèmatiques, 
automatiques et non invasives à la collection de données usuelles déterminantes dans chacun des thèmes abordés.
Les différentes travaux déjà effectués autours de ce sujet ne visaient qu’un seul type de problème à la fois, 
ne permettant pas une vision globale des difficultées rencontrées par les abeilles. Notre but est de réunir les différentes
données qui peuvent être utilisés pour étudier l’influence des éléments et événements extérieurs sur leur santé et leur cadre de vie.\\

Concrétement, l'équipe de SuperBeeLive va concevoir une ruche plate afin d'y mettre en place plusieurs type de capteurs
(hygrométrie, vibrations, température interne et externe, etc) ainsi que des caméras qui filmeront l'intérieur et 
l'extérieur de la ruche. \\

%Schéma prévisionnel de la ruche, photos TODO

Cette instrumentation nous permettra dans un premier temps d'observer les abeilles afin de visualiser leurs comportements 
(danse, regroupement, protection, etc) ainsi que les parasites comme les frelons asiatiques ou les varroas. %TODO lien doc varoa/frelon 
Une fois ces observations faites et documentées, il y aura assez de matière pour pouvoir créer des algorithmes qui reconnaitront 
automatiquement ces comportements pour en sortir des informations spécifiques.\\
Par exemple, un des buts et de repérer la danse d'une abeille qui permet aux autre d'indiquer où se trouve du polen et d'en
extraire les informations qu'elle transmet (la direction à prendre, le temps à parcourir pour y arriver etc).\\
%lien avec document sur les danses ? TODO
Avec ces recherches, il serait possible de mettre en place une vitrine web des caméras streamées et commentées 
automatiquement en temps réel, permettant à d'autres chercheurs d'avoir une ressource permanente pour travailler mais aussi
au grand public d'avoir un accès plus restreint mais pédagogique sur le comportement des abeilles. \\

\subsubsection{Mon rôle durant le stage}
Pour ce stage, plusieurs missions m'ont été affectées, une qui est devenue ma principales et d'autres, plus courtes dans le temps.\\

Comme dit plus haut, le but premier de l'installation des caméras sur la ruche est de pouvoir observer et annoter des vidéos manuellement
pour ensuite rendre cette tâche automatique. Seulement, ces annotations doivent être encadrées afin d'éviter que des données se perdent, 
et que les outils utilisés pour le faire ne soient par les même d'une personne à une autre nous donnant alors des fichiers non 
uniformes et plus difficiles et long à traités une fois réunis. \\
Ainsi, nous avons entreprit de créer un logiciel d'annotation, permettant de regarder en direct les caméras et de sauvegarder des 
morceaux de vidéos afin de pouvoir dessiner simplement dessus (encercler, mettre une flèche, encadrer, etc) et écrire quelques mots
sur ce qu'on y observe. Ces vidéos seraient sauvegardées dans un fichier contenant la vidéo et les annotations et pourront être 
visualisés de nouveau dans ledit logiciel mais aussi dans un lecteur plus classique mais sans les annotations. \\
Au final, celui-ci alégera le travail des biologiste, qui auront un outils sur mesure pour annoter les vidéos, mais aussi de M. Druon, 
pour qui il sera plus simple de récupérer les données.\\

Il est évident que dans un tel projet, beaucoup de données seront transmises et stockées. Ainsi, toute une partie d'administration 
système est à gérer. Dans notre cas, nous avons deux tâches importantes.\\
D'abord, la question du stockage des données commençait à se poser lors de mon arrivée en stage. Il fallait choisir, acheter, installer
puis configurer un serveur de stockage dans la salle serveur du CNRS.\\
Ensuite, le rucher du CNRS où notre ruche expérimentale est installé n'a pas de configuration réseau déjà établie. 
Une connexion par fibre optique est prévue, mais la suite de l'installation devra être gérée par nous même. Comme pour le serveur, 
il faudra choisir, installer et configurer un switch dans le rucher.\\
Pour ces deux tâches, la même problématique est soulevée : il faudra penser au grand nombre de données qui devront transiter sur le 
réseau et donc prévoir du matériel adapté.\\

D'autres réalisations auraient pû m'être affectées, comme le dimensionnement des caméras ou la construction des cartes éléctroniques
qui seront installées au centre de la ruche. Seulement, ces sujets s'éloignant de mon DUT Réseaux et Télécommunications et mes 
compétences et connaissances dans ces deux domaines étant limitées, je n'ai pas eu à travailler dessus.\\
Cependant, la possibilité d'un apprentissage lors d'école d'ingénieurs en systèmes embarqués a été évoquée, ce qui correspondrait 
plus à ces tâches.\\


\subsection{L'équipe}
%TODO organigrame 

Brouillon organigrame 

CNRS : Jean-Baptiste THIBAUD 50\%
IBMM : Matthieu ROUSSET 50\%
LMGC : Delphine JULIEN
       Capucine CARLIER
LIRMM : Jean TRIBOULET 
        Sébastien DRUON 
IUT de Béziers : Philiphe PUJAS -> stockage des données 
Equipe Bee@Wur Université Wegeningen (Néerlandais) -> ouverture internationnalle. 


\section{Projet principal}

    \subsection{Analyse du besoin} 
Dans le cadre de la récolte et de l'analyse des données, il était primordial de réunir des vidéos 

    \subsection{Contraintes} 
    \subsection{Création de l'interface}
        \subsubsection{Principe de GTK}
        \subsubsection{Déclaration}
        \subsubsection{Placement, cosmétique et affichage}
        \subsubsection{Signaux et fonctions}
    \subsection{Traitement vidéo}

\section{Acquis et compétences}
    \subsection{Missions annexes}
    \subsection{Compétences développées}

\section{Conclusion}

\section{Références}
    \subsection{Annexes}
    \subsection{Biblio}
    \subsection{Lexique}

\end{document}
